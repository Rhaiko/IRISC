$  $\pagebreak
\subsection{Software Design}

\subsubsection{Purpose}

The purpose of the On-Board Software consists of:
\begin{itemize}
	\item Controlling the selecting and tracking of targets to observe.
	\item Ensuring that the camera is not oriented towards the sun.
	\item Reading data from sensors and controlling actuators when needed.
	\item Processing and storing images taken by camera.
	\item Logging housekeeping data.
	\item When possible, send images and housekeeping data to ground station.
\end{itemize}

\hl{The software is designed such that it can control the experiment autonomously throughout the whole experiment process but also enable control from ground through telecommands.}

\subsubsection{Design}

\paragraph{a)} Process Overview

\begin{figure}[H]
	\centering
	\includegraphics[width=\textwidth]{4-experiment-design/img/software/process-overview.png}
	\caption{Relations between On-Board Computer and connected components.}
	\label{fig:software-process-overview}
\end{figure}

All external components connected to the On-Board Computer and their interface are displayed in figure \ref{fig:software-process-overview}. \hl{The sanity camera provides an overview of the experiment and captures images on demand. The guiding camera provides an overall view in the direction the telescope is pointing and is also used as a star tracker to determine the gondola attitude. If necessary a secondary microcontroller will be added to support the high sampling frequency needed for the gyroscopes.}

\paragraph{b)} General and safety related concepts

To ensure that the software is not erroneous rigorous testing will be done during development and after completion. A watchdog timer will be used to avoid software freezing. This timer will reset the On-Board Computer if it is not itself reset by the software within a certain period.

\paragraph{c)} Interfaces

If connection to ground is available, images compressed with a lossless compression will be sent down over the E-link over the course of the experiment. If the storage were to fail on touchdown for any reason, this would mean that not all data is lost.

\begin{table}[H]
	\centering
	\begin{tabular}{l|l}
		\textbf{Component}
		& \textbf{Interface} \\ \hline
		Ground Station
		& E-link             \\
		External Data Storage
		& USB            	 \\
		NIR Camera
		& USB                \\
		Guiding Camera
		& USB                \\
		Sanity Camera
		& CSI                \\
		Watchdog
		& Digital Interrupts \\
		Gondola GPS
		& SPI                \\
		Compass/Magnetometer
		& SPI                \\
		Gondola Gyroscope
		& I2C                \\
		Camera Encoders Gyroscope
		& I2C                \\
		Controller Actuators
		& I2C                \\
		Thermal Sensors
		& I2C				 \\
		Heaters \& Coolers
		& I2C
	\end{tabular}
	\caption{Table showing the interface that each external component is connected with. This is also visually represented in figure \ref{fig:software-process-overview}. \hl{[Updated figure]}}
	\label{tab:software-interfaces}
\end{table}


\paragraph{d)} Data acquisition and storage

The main bulk of data handled is the images taken by the camera\hl{s}. Housekeeping data such as positioning, camera direction, time, etc will also be stored along with the images. \hl{The On-Board Computer has one SD card used for the OS and temporary storage. For the external data storage two SD cards are used for redundancy purposes. All data shall be saved on these external SD-cards.}

As the \hl{NIR} camera has a resolution of $5496 * 3672$ and a colour depth of 12\,bits, the raw image size will be 30.27\,MB. For a 4\,hour float phase and an exposure time of 30\,s a total of  14.53\,GB is required in on-board data storage for the images. \hl{An exposure time of 30\,s can't be guaranteed e.g. due to movements of the gondola. If the exposure time is dropped to 10\,s the total amount of data for continuous image capturing is  43.6\,GB.}

\hl{The guiding camera has a resolution of $1936 * 1096$ and a colour depth of 12\,bits leading to a raw image size of 3.18\,MB. If the guiding camera is sampled every 10\,seconds the total data required for the images is 4.58\,GB.}

\hl{As the housekeeping data is negligible in size compared to the images and as the sanity images are only taken on demand, an SD-card size of 64\,GB will be sufficient for all data storage.}

With a maximum of 1\,Mbit per second data transfer rate to ground it takes at least 145.5\,s to transfer one image compressed losslessly to 60\,\% of original size. This means that with little down time between observations and for most exposure times, not all images can be transmitted to ground.

\paragraph{e)} Process Flow

Pre-launch tests shall be conducted to ensure that all systems work as expected. Afterwards the system shall enter a sleep mode\hl{, a low activity state,} with the camera in a safe launch position. \hl{During the ascent the thermal control system shall be running to ensure that all components of the experiment are within nominal temperature ranges.}

When the float phase is reached, the system will wake. \hl{This is done by tracking the altitude of the gondola using GPS data.} After wake-up the system shall find its orientation and the position of the sun. Finally tracking and observation can start.

Astronomical targets are prioritised. The software shall track and observe the highest priority target within the field of view with varying camera settings until one of the following events happen:

\begin{itemize}
	\item Current target leaves operational field of view.
	\item Target moves too close to the sun for observation.
	\item A higher priority target enters the field of view.
\end{itemize}

If one of the aforementioned events happens, the software will switch current target following prioritisation. While observing targets, the On-Board Software shall store images and housekeeping data. If connection to ground is available this data shall be compressed using a lossless compression method and sent to ground.

At the end of the floating phase the camera shall be oriented in a landing position and the system shall shut down. Figure \ref{fig:software-activity-diagram} shows the complete process flow. Figure \ref{fig:software-state-diagram} shows a simple state diagram for the experiment. Observations are done in the Normal state. In the rare case of a software freeze, it will be reset without entering sleep mode.

\begin{figure}[H]
    \centering
    \includegraphics[width=.5\textwidth]{4-experiment-design/img/software/activity-diagram.png}
    \caption{Activity diagram describing the complete software flow. \hl{Here SOE refers to when the software starts, i.e. on the launch pad, and EOE refers to when it shuts down.} \hl{[updated figure]}}
    \label{fig:software-activity-diagram}
\end{figure}

\begin{figure}[H]
	\centering
	\includegraphics[width=.5\textwidth]{4-experiment-design/img/software/state-diagram.png}
	\caption{State diagram for On-Board Software. \hl{[updated figure]}}
	\label{fig:software-state-diagram}
\end{figure}

\clearpage
\paragraph{f)} Modularisation and pseudo code

\begin{figure}[H]
	\centering
	\includegraphics[width=\textwidth]{4-experiment-design/img/software/composition-tree.png}
	\caption{Composition tree of On-Board Software. \hl{[updated figure]}}
	\label{fig:software-composition-tree}
\end{figure}

Figure \ref{fig:software-composition-tree} shows how the complete software is modularised. Each component is described below.

\begin{itemize}

	\item Camera: Parent
		\begin{itemize}
			\item Camera Control: Module responsible for selecting target, camera settings and capturing images.
			\item Guiding Camera: Communication link to the guiding camera.
			\item NIR Camera: Communication link to the NIR camera.
			\item Sanity Camera: Communication link to the sanity camera.
		\end{itemize}

	\item Command: Module responsible for handling incoming commands from ground.

	\item \hl{Data Storage: Module providing an interface to store data to the external SD-cards.}

	\item E-link: Module responsible for communications over the E-link interface.

	\item I2C: Module responsible for communications over the I2C bus.

	\item Img Processing: Image processing, parent
		\begin{itemize}
			\item Data Queue: Buffer to hold camera data until Handler is ready
			\item Image Handler: Module processing and storing images taken by camera.
		\end{itemize}

	\item Init: Module initialising each component.

	\item Mode: Module responsible for the current state of the software.

	\item Sensors: Parent
		\begin{itemize}
			\item Orientation: Module keeping track of orientation and location.
			\item Sun: Module keeping track of position of the sun.
			\item \hl{Sensor Poller: Module responsible for polling sensors.}
			\item Temperature: Thermal sensors.
		\end{itemize}

	\item SPI: Module responsible for communications over the SPI bus.

	\item Telemetry: \hl{Parent}
		\begin{itemize}
			\item \hl{Downlink: Module responsible for sending telemetry to ground.}
			\item \hl{Downlink Queue: Buffer to hold telemetry messages.}
		\end{itemize}

	\item Thermal: Module responsible for active thermal control

	\item Tracking: Parent
		\begin{itemize}
			\item Current Target: Module holding the current target to be observed.
			\item Controller: Module responsible for keeping camera on target.
			\item \hl{Gimbal: Module providing an interface to the gimbal motors.}
			\item Target Choosing: Module responsible for keeping track of target prioritisation.
		\end{itemize}

	\item Watchdog: Timer to reset external watchdog.

\end{itemize}

\subsubsection{Implementation}

The code for the On-Board Software shall be implemented in C. An operating system will be used to enable the modularisation required. \hl{The chosen operating system is Linux with the Preempt RT kernel patch.} Additional libraries\hl{, such as the ASI SDK library used for controlling the cameras, will} \st{may} be needed.

\subsubsection{Control system}
The main task of the control system is selecting and tracking the astronomical targets and stabilising the telescope during exposure. Other tasks include minor tasks like the thermal control of the CMOS sensor and the electronics box as well as control of the actuators.

\paragraph{Selecting and tracking targets}

The selection and tracking of targets is done by using the current time, position (GPS) and orientation \st{(compass/magnetometer)} \hl{(sun sensor/star tracker and gyroscope)}
of the gondola. Once a suitable target in the operational field of view is selected, it is tracked during exposure. This includes the compensation of the following motions:
\begin{itemize}
	\item Time-dependent rotational motion of the astronomical targets in the sky. This will be continuously compensated during exposure using models and/or interpolation of tables.
	\item Position-dependent rotational motion of the astronomical targets in the sky. This will be corrected once for every picture, using the GPS data.
	\item Rotation of the gondola in the z-axis; can be corrected during exposure using a gyroscope sensor.
\end{itemize}

Due to the nature of the motion of astronomical targets, it is necessary to use a 3-axis gimbal.

\textbf{\hl{Selection of targets}}

\hl{The selection of targets is based on the operational field of view (the area of the sky where the telescope is able to look at) and prioritisation parameters of the possible targets within this field of view. The operational field of view is determined by using the sensor data from the orientation sensor (sun sensor/star tracker) in combination with the gyroscopes.

Prioritisation parameters include the brightness of the object (brighter objects are expected to yield better results due to higher SNR), the location of the object within the operational field of view (objects near the centre of the field of view are favoured because they are less likely to rotate out of the field of view during exposure) and the number of exposures already taken (objects should be imaged more than once, but not more often than ten times during the flight). 

As the location of the astronomical targets changes with time and position of the gondola, these two parameters also need to be taken into account (for determination see paragraph Astronomical targets).
}

\textbf{\hl{Astronomical targets}}

\hl{The time-dependent and the position-dependent rotational motion of the astronomical targets are calculated by using an astronomical model and provides the control input for the subsequent stages of the control system. Because the state vectors for this model are time (internal clock) and position (GPS) and the input vector is the selected target (defined in the system, not a variable state for the exposure of one image), the state vectors cannot be affected by the control system (not controllable). Therefore, no dedicated feedback loop is used for this part of the control system. However, the input data from the GPS may be filtered before use. The output vector then provides the input for the compensation of rotation (of the gondola in the z-axis) and the stabilisation of the gimbal. }

\textbf{\hl{Rotation of the gondola}}

\hl{The uncontrolled rotational motion of the gondola around the z-axis can be directly compensated by actuating the corresponding gimbal axis (yaw). This will be achieved by a PID control. As the same axis will also be actuated during the stabilisation process, it might be merged with the control loop for the stabilisation. As the required accuracy for detecting the rotation of the gondola is very high, the gyroscopes are used to provide the input.}


\paragraph{Stabilisation of the gimbal}
The stabilisation of the gimbal only needs to be active during exposure in order to avoid blurred pictures. It is responsible for compensating all kinds of small-scale, unpredicted movements of the gondola. In order to achieve this, an active feedback loop that requires information about the gondola movements is needed. This information is gathered by using accelerometers and gyroscopes for all 3 axes.


\raggedbottom
