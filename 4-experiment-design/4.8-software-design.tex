\pagebreak
\subsection{Software Design}

\subsubsection{Purpose}

The purpose of the on-board software consists of:
\begin{itemize}
	\item Controlling the tracking and choosing of targets to observe.
	\item Ensuring that the camera is not oriented towards the sun.
	\item Process and store images taken by camera.
	\item Log housekeeping data.
	\item When possible, send images and housekeeping data to ground station.
\end{itemize}



\subsubsection{Design}

\paragraph{a)} Process Overview\\

\begin{figure}[H]
	\centering
	\includegraphics[width=\textwidth]{4-experiment-design/img/software/process-overview.png}
	\caption{Relations between on board computer and connected components.}
	\label{fig:software-process-overview}
\end{figure}

All external components connected to the on board computer and their interface are displayed in figure \ref{fig:software-process-overview}. 

\paragraph{b)} General and safety related concepts\\

To ensure that the software is not erroneous rigorous testing will be done during development and after completion. A watchdog timer will be used to avoid software freezing. This timer will reset the on-board computer if it is not itself reset by the software within a certain period. In the case of a reset, the camera shall move to the launch position and tracking restarted.

\paragraph{c)} Interfaces\\

If connection to ground is available, images compressed with a lossless compression will be sent down over the E-link over the course of the experiment. If the storage were to fail on touchdown for any reason, this would mean that not all data is lost.


\begin{table}[H]
	\centering
	\begin{tabular}{l|l}
		\textbf{Component}
		& \textbf{Interface} \\ \hline
		Ground Station
		& E-link             \\
		External Data Storage
		& USB            	 \\
		NIR Camera
		& USB                \\
		Guiding Camera
		& USB                \\
		Sanity Camera
		& CSI                \\
		Watchdog
		& Digital Interrupts \\
		Gondola GPS
		& SPI                \\
		Compass/Magnetometer
		& SPI                \\
		Gondola Gyroscope
		& I2C                \\
		Camera Encoders Gyroscope
		& I2C                \\
		Controller Actuators
		& I2C                \\
		Thermal Sensors
		& I2C				 \\
		Heaters \& Coolers
		& I2C
	\end{tabular}
	\caption{Table showing the interface that each external component is connected with. This is also visually represented in figure \ref{fig:software-process-overview}. \hl{[Updated figure]}}
	\label{tab:software-interfaces}
\end{table}



\paragraph{d)} Data acquisition and storage\\

The main bulk of data handled is the images taken by the camera. Housekeeping data such as positioning, camera direction, time, etc will also be stored along with the images.

%need amount of data

\paragraph{e)} Process Flow\\

Pre-launch tests shall be conducted to ensure that all systems work as expected. Afterwards the system shall enter a sleep mode with the camera in a safe launch position. When the float phase is reached, the system will wake. After wake-up the system shall find its orientation and the position of the sun. Finally tracking and observation can start.

Astronomical targets are prioritised. The software shall track and observe the highest priority target within the field of view with varying camera settings until one of the following events happen:

\begin{itemize}
	\item Current target leaves operational field of view.
	\item Target moves too close to the sun for observation.
	\item A higher priority target enters the field of view.
\end{itemize}

If one of the aforementioned events happen, the software will switch current target following prioritisation. While observing targets, the on-board software shall store images and housekeeping data. If connection to ground is available this data shall be compressed using a lossless compression method and sent to ground.

At the end of the floating phase the camera shall be oriented in a landing position and the system shall shut down. Figure \ref{fig:software-activity-diagram} shows the complete process flow.

\begin{figure}[H]
    \centering
    \includegraphics[width=.5\textwidth]{4-experiment-design/img/software/activity-diagram.png}
    \caption{Activity diagram describing the complete software flow.}
    \label{fig:software-activity-diagram}
\end{figure}

\newpage
\paragraph{f)} Modularisation and pseudo code\\

\begin{figure}[H]
	\centering
	\includegraphics[width=\textwidth]{4-experiment-design/img/software/composition-tree.png}
	\caption{Composition tree of on board software.}
	\label{fig:software-composition-tree}
\end{figure}

Figure \ref{fig:software-composition-tree} shows how the complete software is modularised. Each component is described below.

\begin{itemize}
	\item Img Processing: Image processing, parent
		\begin{itemize}
			\item Image Handler: Module processing and storing images taken by camera.
			\item Data Queue: Buffer to hold camera data until Handler is ready
		\end{itemize}
	\item Thermal: Module responsible for active thermal control
	\item Camera: Parent
		\begin{itemize}
			\item Camera Control: Module responsible for selecting target, camera settings and capturing images.
			\item Data Link: Communication link to camera.
		\end{itemize}
	\item Watchdog: Timer to reset external watchdog.
	\item Init: Module initialising each component.
	\item Tracking: parent
		\begin{itemize}
			\item Current Target: Module holding the current target to be observed.
			\item Controller: Module responsible for keeping camera on target.
			\item Target Choosing: Module responsible for keeping track of target prioritisation.
		\end{itemize}
	\item Telemetry: Module responsible for sending telemetry to ground.
	\item Sensors: parent
		\begin{itemize}
			\item I2C: Module responsible for communications over the I$^2$C bus.
			\item Temperature: Thermal sensors.
			\item Camera Orientation: Module keeping track of camera orientation and location.
			\item Sun: Module keeping track of position of the sun.
			\item Sanity Camera: Connection to the small sanity camera.
			\item SPI: Module responsible for communications over the SPI bus.
		\end{itemize}
		
\end{itemize}

\subsubsection{Implementation}

The code for the on-board software shall be implemented in C. An operating system will be used to enable the modularisation required.

\raggedbottom
