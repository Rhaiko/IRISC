\pagebreak
\subsection{Power System}

\label{sec:4.7}

A 28.8V (13 Ah) battery package can be provided by the gondola, according to the BEXUS user manual. One constraint is that the continuous maximum current is 1.8 A. Thus buck-converters will be needed to step down the voltage, while stepping up the current, so that the power needed will be delivered to the instrument. The voltage levels of the buck-converters are:

\begin{itemize}
	\item 28.8 V --> 3.3 V, with the power 1.65 W.
	\item 28.8 V --> 5 V, with power 15 W.
	\item 28.8 V --> 12 V, with power 15 W.
\end{itemize}

A schematic of the power system can be found in the appendix C.

The estimated power consumption of the components are shown in table \ref{tab: power consumption}.
% import table of estimated power for components.

% highy recommend https://www.tablesgenerator.com/

\begin{center}
\begin{table}[H]
\begin{tabular}{|m{0.2\textwidth}|m{0.1\textwidth}|m{0.25\textwidth}|m{0.15\textwidth}|m{0.1\textwidth}|}
\hline
\textbf{Component Name} & \textbf{Supply Voltage {[}V{]}} & \textbf{Input Current {[}A{]} (MAX)} & \textbf{Power {[}W{]} (MAX)} & \textbf{Quantity} \\ \hline
Raspberry Pi CM3 Lite   & 5                               & 2                                   & 10                           & 1                 \\ \hline
DC Motors               & 12                              & 0.35                                & 4.2                          & 3                 \\ \hline
Gyroscope               & 3.3                             & 6.1m                                & 0.02013                      & 2                 \\ \hline
Magnetometer            & 3.3                             & 8m                                  & 0.0264                     & 1                 \\ \hline
GPS                     & 3.3                             & 10m                                 & 0.033                        & 1                 \\ \hline
Encoders                & 5                               & 8m                                  & 0.04                         & 2                 \\ \hline
Heater Motors           & 5                               & 0.188                               & 0.94                         & 3                 \\ \hline
Heater Camera           & 5                               & 0.382                               & 1.91                         & 1                 \\ \hline
Heater Electronics      & 5                               & 0.19                                & 0.95                         & 1                 \\ \hline
Camera                  & 5                               & 0.3                                 & 1.5                          & 1                 \\ \hline
Sanity Camera           & 3.3                             & 0.25                                & 0.825                        & 1                 \\ \hline
Guiding Camera          & 5                               & 0.3                                 & 1.5                          & 1                 \\ \hline
Accelerometer           & 3.3                             & 0.000155                            & 0.000558                     & 1                 \\ \hline
Temperature sensors     & 3.3                             & 1.5m                                & 0.00495                      & 5                 \\ \hline
Buck Converter 3.3V     & 28.8                            & 0.5(output)                         & 0.18333                      & 1                 \\ \hline
Buck Converter 5V       & 28.8                            & 3(output)                           & 1.6667                       & 1                 \\ \hline
Buck Converter 12V      & 28.8                            & 1.25(output)                        & 3.75                         & 1                 \\ \hline
\textbf{TOTAL}          & \textbf{-}                      & \textbf{4.946855}                   & \textbf{38.0099545}           &                   \\ \hline
\end{tabular}
\caption{The estimated power consumption of the components in active mode.}
\end{table}
\label{tab: power consumption}
\end{center}




\raggedbottom

\newpage

The BEXUS manual recommends the instrument be prepared to have power supplies for 2 hours of testing, 2 hours on ground and for a flight time of 6 hours as a minimum. The instrument therefore needs to be in active mode for at least 8 hours, because on ground the instrument will be in sleep mode. The instrument will also be in sleep mode for ascending and descending of the gondola. 

The total estimated power for the instrument:

\begin{itemize}
    \item Active mode: 38.02 W  --> 304 Wh (8h)
    \item Sleep mode: 15 W --> 45Wh (3h)
\end{itemize}

Thus the instrument will require at least 349 Wh. The maximum power provided by the gondola will be 375 Wh, which should satisfy the instruments needs.


\raggedbottom
