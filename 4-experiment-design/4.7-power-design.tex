\pagebreak
\subsection{Power System}

\label{sec:4.7}

A 28.8V (13 Ah) battery package can be provided by the gondola, according to the BEXUS user manual. \hl{Thus the gondola would provide 374.6 Wh. But due to the harsh environment in the gondola one should expect a capacity no larger than 6 Ah and a cell voltage around 3 V, which means that the gondola would provide 144 Wh. One constraint is that the batteries has a 5A fuse, and a continuous maximum current is 1.8 A is recommended.}Thus Buck-converters will be needed to step down the voltage, while stepping up the current, so that the power needed will be delivered to the instrument. The voltage levels of the buck-converters are:

\begin{itemize}
	\item 28.8 V --> 3.3 V, with the power 1.65 W.
	\item 28.8 V --> 5 V, with power 15 W.
	\item 28.8 V --> 12 V, with power 15 W.
\end{itemize}


The estimated power consumption of the components are shown in table \ref{tab: power consumption}.
% import table of estimated power for components.

% highy recommend https://www.tablesgenerator.com/

\begin{center}
\begin{table}[H]
\begin{tabular}{|m{0.2\textwidth}|m{0.1\textwidth}|m{0.25\textwidth}|m{0.15\textwidth}|m{0.1\textwidth}|}
\hline
\textbf{Component Name} & \textbf{Supply Voltage {[}V{]}} & \textbf{Input Current {[}A{]} (MAX)} & \textbf{Power {[}W{]} (MAX)} & \textbf{Quantity} \\ \hline
Raspberry Pi CM3 Lite   & 5                               & 2                                   & 10                           & 1                 \\ \hline
DC Motors               & 12                              & 0.35                                & 4.2                          & 3                 \\ \hline
Gyroscope               & 3.3                             & 6.1m                                & 0.02013                      & 2                 \\ \hline
Magnetometer            & 3.3                             & 8m                                  & 0.0264                     & 1                 \\ \hline
GPS                     & 3.3                             & 10m                                 & 0.033                        & 1                 \\ \hline
Encoders                & 5                               & 8m                                  & 0.04                         & 2                 \\ \hline
Heater Motors           & 5                               & 0.188                               & 0.94                         & 3                 \\ \hline
Heater Camera           & 5                               & 0.382                               & 1.91                         & 1                 \\ \hline
Heater Electronics      & 5                               & 0.19                                & 0.95                         & 1                 \\ \hline
Camera                  & 5                               & 0.3                                 & 1.5                          & 1                 \\ \hline
Sanity Camera           & 3.3                             & 0.25                                & 0.825                        & 1                 \\ \hline
Guiding Camera          & 5                               & 0.3                                 & 1.5                          & 1                 \\ \hline
Accelerometer           & 3.3                             & 0.000155                            & 0.000558                     & 1                 \\ \hline
Temperature sensors     & 3.3                             & 1.5m                                & 0.00495                      & 5                 \\ \hline
Buck Converter 3.3V     & 28.8                            & 0.5(output)                         & 0.18333                      & 1                 \\ \hline
Buck Converter 5V       & 28.8                            & 3(output)                           & 1.6667                       & 1                 \\ \hline
Buck Converter 12V      & 28.8                            & 1.25(output)                        & 3.75                         & 1                 \\ \hline
\textbf{TOTAL}          & \textbf{-}                      & \textbf{4.946855}                   & \textbf{38.0099545}           &                   \\ \hline
\end{tabular}
\caption{The estimated power consumption of the components in active mode.}
\end{table}
\label{tab: power consumption}
\end{center}




\raggedbottom


The BEXUS manual recommends the instrument be prepared to have power supplies for 2 hours of testing, 2 hours on ground and for a flight time of 6 hours as a minimum. \hl{The experiment will be in active mode only during float, which should be no longer than 4 hours, but on ground and during ascend and descend the experiment will be in stand by mode, for about 6 hours. The power dissipation for the different modes are listed below:}


\begin{itemize}
    \item \hl{Active mode : 41.15 W --> 165 Wh (4h) }
    \item \hl{Standby mode: 14 W --> 84Wh (6h)}
\end{itemize}

\hl{In total the experiment in total for 10 hours would need 249 Wh. Thus the experiment needs two battery packages. With two battery packages the gondola would be able to provide at least 288 Wh, which should satisfy the power needs of the experiment. }



\raggedbottom
