\subsection{Performance Requirements}

\begin{enumerate}
    \item[P.1] The gimbal stabilization system shall point the telescope towards the celestial body with an accuracy of at least 1 arc seconds.
    \item[P.2] The optics shall be cable of making pictures of 0.5-1.5 x 0.3-1\,degrees.
    \item[P.3] The NIR camera shall make images in the range of 720-850 to 1200\,nm.
    \item[P.4] The NIR camera shall have a resolution of at least 16\,MP.
    \item[P.5] The NIR camera shall be able to make images with exposure times between 0.5 and 150\,seconds.
    \item[P.6] The experiment shall establish the location and orientation of the gondola.
\end{enumerate}