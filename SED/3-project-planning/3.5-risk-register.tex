\pagebreak
\subsection{Risk Register}
\textbf{Risk ID}
\begin{enumerate}[label={}]
    \item TC – Technical/Implementation 
    \item MS – Mission (operational performance) 
    \item SF – Safety 
    \item VE – Vehicle 
    \item PE – Personnel 
    \item EN – Environmental 
    \item OR - Outreach
    \item BG - Budget
\end{enumerate}

Adapt these to the experiment and add other categories. 
Consider risks to the experiment, to the vehicle and to personnel. 

\textbf{Probability (P)}
\begin{enumerate}[label=\Alph*]
    \item Minimum – Almost impossible to occur 
    \item Low – Small chance to occur 
    \item Medium – Reasonable chance to occur 
    \item High – Quite likely to occur 
    \item Maximum – Certain to occur, maybe more than once
\end{enumerate}

\textbf{Severity (S)}
\begin{enumerate}
    \item Negligible – Minimal or no impact 
    \item Significant – Leads to reduced experiment performance 
    \item Major – Leads to failure of subsystem or loss of flight data 
    \item Critical – Leads to experiment failure or creates minor health hazards 
    \item Catastrophic – Leads to termination of the REXUS/BEXUS programme, damage to the vehicle or injury to personnel 
\end{enumerate}

The rankings for probability (P) and severity (S) are combined to assess the overall risk classification, ranging from very low to very high and being coloured green, yellow, orange or red according to the SED guidelines.

Whether a risk is acceptable or unacceptable has been assigned according to the SED guidelines. Where mitigation is written for acceptable risks this details the mitigation undertaken in order to reduce the risk to an acceptable level.

\begin{landscape}
% highy recommend https://www.tablesgenerator.com/


\begin{longtable}{|m{0.075\textwidth}| m{0.48\textwidth} |m{0.02\textwidth} |m{0.02\textwidth}|m{0.10\textwidth}| m{0.64\textwidth}|}

\hline
\textbf{ID} & \textbf{Risk (\& consequence if)} & \textbf{P} & \textbf{S} & \textbf{P * S} & \textbf{Action} \\ \hline


TC10 & Optics and/ or camera destroyed due to testing						& C & 2 & \cellcolor[HTML]{FCFF2F}Low 		& There is budget for a spare part, it is quite easy to get and test where it will likely fail (e.g. drop test) will not be done with this part.\\\hline

TC20 & Optics and/ or camera destroyed due to looking directly into the sun & B & 3 & \cellcolor[HTML]{FCFF2F}Low 		& A model will be made and \hl{thorough testing will be made to mitigate this risk. The control system will also ensure to avoid looking into the sun }.\\\hline

TC30 & Software failure														& B & 3 & \cellcolor[HTML]{FCFF2F}Low 		& A watchdog with power-on-reset will be added to the design.\\\hline

TC40 & Motors of the gimbal are uncontrollable								& B & 4 & \cellcolor[HTML]{FCFF2F}Low 		& It will be made sure that a single component failure will not result in this consequence.\\\hline

TC50 & Motors overloaded													& A & 3 & \cellcolor[HTML]{34FF34}Very Low 	& Sufficient testing and modeling will be done to decrease the probability.\hl{Safety measures will be added to the power system to prevent overloading}\\\hline

TC60 & PCB failure \hl{due to defective traces or component}												& B & 3 & \cellcolor[HTML]{FCFF2F}Low 		& \st{sufficient} Testing will be performed on each PCB to decrease the probability.\\\hline

TC70 & Single component failure gives unprecedented failure					& B & 4 & \cellcolor[HTML]{FCFF2F}Low			& A Failure mode and effects analysis (FMEA) study will be done so that single failure components with a high impact will be documented and migrated.\\\hline

TC80& Moving parts jamming \hl{due to moisture freezing}& B & 3 & \cellcolor[HTML]{FCFF2F} Low & Moving parts will be \st{properly} cleaned and lubricated where needed. \\\hline
\hl{TC90} & \hl{The structure is not stiff enough} & \hl{A} & \hl{2} & \cellcolor[HTML]{34FF34}Very Low & \hl{FEM analysis and testing will performed to ensure the necessary stiffness is achieved.} \\ \hline

\hl{TC90} & \hl{Values for R and Q matrices of PID controller not chosen correctly for Kalman filter} & \hl{B} & \hl{2} & \cellcolor[HTML]{FCFF2F}Low	& \hl{Thorough analysis, simulations and testing will be conducted to tune the tuned correctly.} \\ \hline

MS10 & Target not found	& D & 2 & \cellcolor[HTML]{FCFF2F}Low			& The ground station will be able to correct this, or send the system to another target.\\\hline

MS20 & Damage to the system on landing										& D & 1 & \cellcolor[HTML]{FCFF2F}Low			& Optics will rotate into a position where damage can be avoided/minimized.\\\hline

MS30 & Damage to the storage unit on landing									& C & 2 & \cellcolor[HTML]{FCFF2F}Low			& Data will be send over telemetry\\\hline

MS40 & Storage unit full during flight											& A & 3 & \cellcolor[HTML]{34FF34}Very Low	& \hl{The data handling system will be thoroughly tested, and the storage unit(s) will be of more capacity than it is expectede to use} \st{modeling will be done to decrease the probability}.\\\hline

MS50 & BEXUS balloon power failure											& A & 4 & \cellcolor[HTML]{34FF34}Very Low	& -\\\hline

MS60 & BEXUS balloon telemetry failure										& B & 2 & \cellcolor[HTML]{34FF34}Very Low	& The system will be able to function autonomously.\\\hline

MS70 & Critical electronic components overheating & B & 3 & \cellcolor[HTML]{34FF34}Very Low & Peltier elements and heat sinks will be added where necessary. \\\hline

MS80 & Scientific objective not achieved & B & 3 & \cellcolor[HTML]{FCFF2F} Low & \hl{Systems will be designed so that scientifically relevant data will be obtained, for example, sensor fusion will be implemented to the gyroscope and star tracker to improve tracking accuracy.}\st{Critical components for the pointing system will be selected, designed and manufactured with a high enough safety factor.}\\\hline

SF10 & Components falling off the gondola									& A & 4 & \cellcolor[HTML]{34FF34}Very Low	& All parts \hl{securely} \st{sufficiently} fastened. Testing will be conducted to ensure the fastening is able to hold all parts in place \st{in case of turbulence}. Where deemed necessary, additional fixations will be added.\\\hline

VE10 & Short circuiting														& B & 3 & \cellcolor[HTML]{FCFF2F}Low			& A fuse is added in the power system.\\\hline

PE10 & Miscommunication in the team											& B & 2 & \cellcolor[HTML]{34FF34}Very Low	& The management team should be responsible for ensuring \hl{important} \st{proper} information is conveyed to all team members.\\\hline

PE20 & People are not available												& C & 1 & \cellcolor[HTML]{34FF34}Very Low	& An availability sheet is made so a planning can be made. This will be kept up to date during the entire project.\\\hline

PE30 & Management team unavailable to oversee project						& C & 1 & \cellcolor[HTML]{34FF34}Very Low	& There is always someone as a backup.\\\hline

PE40 & Sudden resignation of project members								& B & 2 & \cellcolor[HTML]{34FF34}Very Low	& Management team ensure morale remains high through a variety of techniques (team building activities).\\\hline

\hl{BG10} & \hl{Budget is not sufficient} & \hl{B} & \hl{4} & \cellcolor[HTML]{FCFF2F}Low & \hl{Management team will ensure that the budget is enough as well as keep in contact with companies and institutions for any possible support and/or sponsorship.} \\ \hline

\hl{BG20} & \hl{Schedule delays} & \hl{C} & \hl{2} & \cellcolor[HTML]{F39C12} Medium & \hl{Planning of critical dates will be done and made easily accessible for all the team members} \\ \hline



\caption{Risk Register.}
\label{tab:risk-register}
\end{longtable}
\raggedbottom
\end{landscape}