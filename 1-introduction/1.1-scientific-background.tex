\subsection{Scientific Background}

The universe hides a plethora of information across the entire electromagnetic spectrum. Infrared light is unique as compared to other wavelengths of light in that it can permeate interstellar dust, and corresponds to typically "cold" astronomical phenomena, such as star formation, some nebulae, and older, colder stars, to name a few \cite{Targets}. Here on the surface of the earth, both visible light and radio waves exhibit near-total transmission through the atmosphere for observation, and experience fewer ill effects from atmospheric interference than other wavelengths, including infrared light, that cannot penetrate the atmosphere at all. The infrared regime lies between visible and microwave radiation, and in an astronomical context this usually corresponds to wavelengths between 0.75 and 300 microns. Within this range, a number of cosmological phenomena can be explored, however only parts of this spectral range are capable of penetrating the earth's atmosphere for observation.

Factors affecting infrared transmission through the atmosphere include molecular absorption (in particular absorption by carbon monoxide, carbon dioxide, water vapour, and oxygen, among others), scattering by dust and other molecules, and distortion of the light by small-scale perturbations in the atmosphere such as pressure and temperature gradients \cite{AstroToday}. As with visible light, infrared observations are also not possible on cloudy or overcast nights. To minimise these problems, infrared telescopes are usually situated at very high altitudes, such as the Tokyo Atacama Observatory (TAO) at 5,600\,m above sea level \cite{TAO}. Better still, the Stratospheric Observatory for Infrared Astronomy (SOFIA) is an airborne observatory that flies at an altitude of 13-14\,km, thus getting above 99.8\% of atmospheric water vapour absorption \cite{SOFIA}, as well as the majority of scattering and distortion. 
