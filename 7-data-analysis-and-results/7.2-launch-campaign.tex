\pagebreak
\subsection{Launch Campaign}
\subsubsection{Flight preparation activities during launch campaign} %.... maybe some of this is currently in chapter 6
The flight preparations can be found in Section \ref{prep_for_Esrange}.

\subsubsection{Flight performance}
\begin{itemize}
	\item Systems were nominal until 12km of altitude where the GPS stopped working completely due to misconfiguration. 
	\item Manual command to move the telescope were sent. At first there was no movement in one of the axes, likely due to a frozen gearbox. This problem went away after a few minutes of moving the telescope via commands
	\item Guiding camera stopped responding
	\item Different exposure times for main camera were set in order to obtain different data from the pictures. During flight, the main camera appeared to work nominally according to the telemetry data
	\item Before the cut-off from the balloon,  the telescope was moved to the landing position.
\end{itemize} %Expected downlink? Expected things that will happen during flight>

\subsubsection{Recovery}
It took four days to recover the gondola from it's landing spot in Finland due to bad weather. Upon impact, the azimuth shaft bent but did not break. Every other component, mechanical or otherwise, remained in good shape. Some rust was present in some of the steel components like the azimuth and elevation shafts, likely due to the snowy conditions at the landing spot. The storing device was in perfect shape and data was successfully recovered.
% Any special requests? Quick recovery? Parts that need to be covered or have something done before transportation?

\subsubsection{Post flight activities}
After the flight, the data was recovered from the on-board SD card. The data was analyzed as discussed in chapter 7.%What will you do with the experiment at Esrange after flight? Mostly initial data analysis.
