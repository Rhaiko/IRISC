\subsection{Data Analysis Plan}

In a best-case scenario, where there are no pointing errors, the CMOS sensor experiences no malfunctions and all astronomical targets are imaged as planned with images delivered in RAW, then the only data analysis to be done is to compare the signal-to-noise ratio (SNR) of the control images to the flight images, and then to images of the given targets in the same wavelength from other instruments, if applicable.  Images to be processed and analysed are expected to be delivered in RAW and at full resolution for a size of 20-30\,MB each, and will therefore require substantial computing power to stack and process.

\subsubsection{Control Images}

A set of control images is necessary to establish a baseline SNR to compare to the flight images. To obtain these control images, the entire apparatus should be functional in order to as closely simulate the BEXUS gondola inertial environment as possible (temperature and pressure conditions do not need to be the same). In order to image the same targets as will be available during the BEXUS campaign, daylight constraints and the positions of the targets require that attempts begin 4-6 weeks before the experimental instrumentation needs to be completed in full. Should this window be missed, imaging attempts after the launch (provided that the instrumentation survive the landing) can continue for another 6-8 weeks before the targets move out of sky view.

\subsubsection{Image Processing and SNR Determination}

In astrophotography, there are four main kinds of images that contribute to a final image:
\begin{itemize}
    \item \textbf{Light Frames:} the image of the target itself, with no modifications or corrections yet applied.
    \item \textbf{Dark Frames:} images taken at the same settings and under the same conditions as the light frames, but with the aperture blocked of light.
    \item \textbf{Flat Frames:} these determine the brightness balance of the sensor-- a lot of sensors are 'brighter' toward the middle of the frame and 'darker' toward the edges.
    \item \textbf{Bias Frames:} these are taken at as fast an exposure time as possible, with the lens cap on, to remove the readout noise due to the sensor.
\end{itemize}

These four kinds of images need to be sequentially divided and subtracted from the light image to remove as much noise as possible from the final image. Quantitative determination of the SNR of each image will occur once all frames have been processed through appropriate software, e.g. SIPS, Deep Sky Stacker or RegiStax.
