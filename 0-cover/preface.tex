\section*{PREFACE} \markboth{}{}
\addcontentsline{toc}{section}{PREFACE}

The Rocket and Balloon Experiments for University Students (REXUS/BEXUS) programme was realized under a bilateral Agency Agreement between the German Aerospace Center (DLR) and the Swedish National Space Agency (SNSA). The Swedish share of the
payload was made available to students from other European countries through a collaboration with the European Space Agency (ESA).

EuroLaunch, a cooperation between the Esrange Space Center of SSC and the Mobile Rocket Base (MORABA) of DLR, was responsible for the campaign management and operations of the launch vehicles. Experts from DLR, SSC, ZARM, and ESA provided
technical support to the student teams throughout the project.

The Student Experiment Documentation (SED) is a continuously updated throughout the duration of the  BEXUS student experiment IRISC - InfraRed Imaging of astronomical targets with a Stabilized Camera and will underwent reviews during the preliminary design review, the critical design review, the integration progress review, and final experiment report.

The goal of the IRISC experiment was to obtain images in the near infrared (NIR) spectrum from astronomical targets. Possible targets included the Andromeda Galaxy, Pinwheel Galaxy, Iris Nebula, Eagle Nebula and Starfish Cluster. The images were obtained using a highly stabilized telescope with NIR camera mounted on a BEXUS balloon. With this balloon-borne telescope most interference caused by the atmosphere was avoided-- a problem for most ground-based telescopes-- while keeping the building and operation costs low, compared to an orbital telescope. The stabilization was achieved by a gimbal-like system, this was needed to obtain high quality images while being on a moving platform. For a NIR telescope, it was also important to keep the temperature as low as possible to avoid heat-induced noise. For example, orbital telescopes are kept at only a few degrees above 0 K. IRISC wanted to use a NIR camera with a higher operating temperature (closer to 273 K) that required a relatively simple cooling system. The aim was to develop a simple astronomical research system that is affordable and readily available for integration with other future stratospheric balloon experiments. 
%The bit specific to your team goes here. Obviously remove the bit about TUBULAR and insert your own name xP
