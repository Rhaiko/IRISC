
\subsection{Outreach Approach}
An important focus of the project was the topic of outreach. During the whole timeline, updates about the project was shared with as many people as possible. The most effective way of reaching people is via social media. IRISC was represented on the following social media platforms:
\begin{itemize}
	\item facebook.com/IRISCBexus
	\item instagram/IRISC\_BEXUS
	\item twitter/IRISCBEXUS
\end{itemize}
Additionally to the social media platforms, the team's website with information about the project was available and updates were blogged there regularly.\newline\newline
Future outreach plans included:
\begin{itemize}
	\item articles at university blog of the Lule{\aa} University of Technology
	\item articles in local newspaper
	\item visibility through posters, flyers, etc at university
	\item presentations during local events
	\item applications to grants and awards
\end{itemize}

In addition to the above outreach goals, the team would like to coordinate community and school visits with the optical part of the experiment for astronomical observing with the general public. With a solar filter, the telescope would be capable of observing the sun, and with the appropriate lenses would be capable of observing the moon and other popular objects, such as the planets, in dark sky conditions. Public astronomy opportunities not only fuel public interest in space, science and engineering, but would give the team the opportunity to raise awareness of the IRISC project and the REXUS/BEXUS project as a whole.\\
Contact was made with the local "space high school" (\textit{Rymdgymnasiet}) here in Kiruna, Northern Sweden, to gauge interest in involvement with the project, with the intention of giving a talk about the project to prospective space science students in the local community.